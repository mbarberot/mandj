\documentclass[11pt]{report} % { book | article | report }
% -------------------- Langue --------------------
\usepackage[latin1]{inputenc}
\usepackage[T1]{fontenc}
\usepackage[english]{babel}
% -------------------- Marges, interlignes--------------------
\usepackage[a4paper]{geometry}	% 
%\usepackage{setspace}
%\usepackage{layout}		% Layout
%\usepackage{lscape}		% Orientation de la page
% -------------------- Soulignement --------------------
%\usepackage{soul}
%\usepackage{ulem}
% -------------------- Fontes --------------------
%\usepackage{eurosym}		% Symbole Euro
%\usepackage{bookman}
%\usepackage{charter}
\usepackage{newcent}
%\usepackage{lmodern}
%\usepackage{mathpazo}					
%\usepackage{mathptmx}
% -------------------- Code et citations --------------------
%\usepackage{verbatim}		% Code
%\usepackage{listings}		% Code avec coloration syntaxique
%\usepackage{url}		% Urls
%\usepackage{moreverb}
% -------------------- Couleur --------------------
%\usepackage{color}		% les couleurs
%\usepackage{colortbl}		% les couleurs dans un tableau
%\usepackage[svgnames]{xcolor}	% le package color, en mieux !
% -------------------- Maths --------------------
%\usepackage{amsmath}
%\usepackage{amssymb}
%\usepackage{mathrsfs}
%\usepackage{asmthm}
% -------------------- Images --------------------
\usepackage{graphicx}		% Insertion d'images ou de pdf -> option : [pdftex]
%\usepackage{wrapfig}		% Insérer des images dans un paragraphe
% -------------------- Autres --------------------
%\usepackage{makeidx}		% Création d'index
%\usepackage{fancyhdr}		% Personnalisation des entetes
%\usepackage{sfheaders}		% Mise sans-serif des titres
%\usepackage{enumitem}		% Personnalisation des environnement enumerate	
%\usepackage{pdfpages}
%\usepackage[nottoc, notlof, notlot]{tocbibind}
%\usepackage{tocvsec2}		% Donne le choix du départ de la numérotation


% -- Avec le package listings
% \lstset{language=java}
% -- Avec le package tocvsec2 (dans le document)
% \maxsecnumdepth{chapter}


% Place des guillemets autour du texte
% \guil{anything}
\newcommand{\guil}[1]{\og #1 \fg}

% Place un texte dans une fbox avec retour à la ligne automatique
% \fboxln{anything}
\newcommand{\fboxln}[1]{%
  \fbox{%
    \begin{minipage}{0.9\textwidth}%
    #1%
    \end{minipage}%
  }
}



% Met un élément en gras
% \strong{anything}
\newcommand{\strong}[1]{\textbf{#1}}

 
\newcommand{\imgpath}[1]{images/#1}


\begin{document}
\begin{titlepage}
\begin{center}
 
\vspace*{1.5cm}
\huge
Java client for comics strips

\vspace{1.5cm}

\LARGE
\begin{center}
  \item Students :
  \item BARBEROT Mathieu - RACENET Joan
\end{center}

\begin{center}
  \item Tutors :
  \item BOUQUET Fabrice - COQBLIN Matthias
\end{center}


\vspace{1cm}
May 9, 2012

\vspace{1cm}
UFR ST - Besan\c{c}on \\
\includegraphics[height=1cm]{\imgpath{logo-ufc.png}}


\end{center}
\end{titlepage}
\newpage

\section*{Abstract}
BDovore is a website focused on comic books that allows collectors to manage their library. For several years, students in computer science have been developing a software in cooperation with that web site.  
\newpage

\section*{Thanks}
For their help, their advises and for having provided this subject, we would like to thank our tutors : Mr BOUQUET et Mr COQBLIN.
\newpage

\tableofcontents
\newpage

\chapter*{Introduction}
Managing a library can be hard, especially for collectors who could possess hundreds of books. That is why websites like BDovore\footnote{www.bdovore.com} have been created. BDovore is specialized in comic books and allows its users to manage their library online. But because we are not always connected to the Internet, UFR ST and BDovore started a project : create a Java software you could use offline on your computer that would provide the same functionalities as the website.


\chapter{The Software}

\section{The features}
\subsection{Searching an album}
As the user will need to find the albums he owns in the software, a research function was necessary. Really basic, it was only using a single keyword and the user had to take care of capital letters and other special characters, as accents. By example, the research ''asterix`` did not give any result, in contrary to ''Astérix`` !\\
The results were displayed in a list with the title, the tome number, the series, the genre and the ISBN number. A double click shows a window with details about the album.\\
If your research had a lot of result, four buttons were available in the bottom of the list to go to the first, previous, next or last page.

\subsection{Add or remove an album oh the user library}
Once the window has appear after a double click, some checkboxes are displayed. To add an album, the user has to check the box named ''Owned``. Then three checkboxes become available for details, which  enable the user to notice if the album is dedicated, lent or if he predict to buy it later.

\subsection{See statistics}
Once the user library is filled up, he can see some statistics on that library. There is two different statistics : the first, on the top, is displaying amounts of album of each type (albums in the library, owned and to buy). The second shows percentages of genres, editors, scriptwriters or cartoonists.

\subsection{Configure the software}
Finally, a configuration tool is provided for setting a proxy server or to connect the user on its website's account (to synchronizing its library with the online one).

\section{Unachieved functionalities}
\subsection{Update the research base}
The research feature is working with a database of all the albums which are listed on the website. To keep it to date, a feature that download all the new albums on the website was planned but previous groups did not have the time to finish it.

\subsection{Synchronizing user account}
As user can suggest albums on the website or change details of the albums on the software, the synchronizing feature had to get the two libraries, search the differences and update them. For ambiguous cases, a window show the differences and asks user to decide whether the online one or the offline one is the more up to date. All the reflections on the feature were done but were not integrated in the software.

\section{Our contribution}
For the time we had to work on this project, our main goal was to finish to put in place the unachieved functionalities. In this purpose, we had to :
\begin{itemize}
\item Change the research tool, to make it more efficient and intuitive, so the user could write its requests like any search engine : the user would be able to make a request with several words in any orders and without caring about accents or syntax. He could also use some special characters and keywords to specify its research (like the ''*`` joker or the double quotes).
\item % da web service
\item % da synchro
\item % small mods

\end{itemize}


\chapter{Discovering the software}

\section{Understanding previous students work}
%
% Understanding how the previous students made the software
% Testing
% Finish the synchronization
%
%
\section{Laying the foundation}
%
% Last year, two groups of student worked on that software
% => 2 different software
% => All in one !
%
%
%
\section{Test and updates}
%
% Tests : 
% search is crappy
% synch does not worked
% wsdl not or badly used
%
% update of the database
% update of the 'search motor'
%

\chapter{Web service and Synchronization}
\section{What is a web service?}
%
% Schema
% A list of functionalities
% Make a layer between the server and the client
%
\section{Why do we need it?}
%
% Simplify the development
% Without it, if the server change, the software has to be updated.
% With it, if the server change, the web service implementation will change in the server and the software will not need an update.
%
\section{The synchronization}
%
% with the research base
% with user's account
% 

\chapter*{Conclusion}
% New functionalities :
% - synch for research base and user account
%
% Abandoned functionalities :
% - nothing
%
% What we have done :
% - a unique software (!= two old software)
% - set up a web service
% - implement it in the software
%
% What we haven't finished
% - tests
%
% New functionalities to developp :
% - manually add a comic strip


\newpage
\section*{Abstract}
BDovore is a website focused on comic books that allows collectors to manage their library. For several years, students in computer science have been developing a software in cooperation with that web site. The principal purpose of this software is to give the same features that the website, but without the obligation of being connected to the Internet. By this fact, the user will be able to handle his library both with the website and the program.
\newline This report is about the work done by students on this software, in 2012.
After a review of the state of the software before their contributions, the present report will focus on these one : first, the reinforcing of some functionalities of the program and then, the synchronization feature (aka the mechanisms to unified the library of the user on the website and in the software).

\section*{Thanks}
For their help, their advises and for having provided this subject, we would like to thank our tutors : Mr BOUQUET et Mr COQBLIN.


\end{document}

