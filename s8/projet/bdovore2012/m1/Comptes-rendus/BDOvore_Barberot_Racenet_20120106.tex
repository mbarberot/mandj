\documentclass[12pt]{article}


\usepackage[utf8]{inputenc}
\usepackage[T1]{fontenc}
\usepackage[francais]{babel}
\usepackage[a4paper]{geometry}
%\usepackage{setspace}
%\usepackage{layout}	
\usepackage{lscape}
%\usepackage{soul}
%\usepackage{ulem}
\usepackage{eurosym}		
%\usepackage{bookman}
%\usepackage{charter}
%\usepackage{newcent}
%\usepackage{lmodern}
\usepackage{mathpazo}					
%\usepackage{mathptmx}
%\usepackage{verbatim}	
%\usepackage{listings}	
%\usepackage{url}		
%\usepackage{moreverb}
%\usepackage{color}	
%\usepackage{colortbl}	
%\usepackage[svgnames]{xcolor}
%\usepackage{amsmath}
%\usepackage{amssymb}
%\usepackage{mathrsfs}
%\usepackage{asmthm}
%\usepackage[pdftex]{graphicx}	
%\usepackage{wrapfig}		
%\usepackage{makeidx}	
%\usepackage{fancyhdr}
%\usepackage{sfheaders}
%\usepackage{pdfpages}
%\usepackage[nottoc, notlof, notlot]{tocbibind}
\usepackage{tocvsec2}


\title{Client Java BDOvore\\Compte rendu de réunion}
\date{6 Janvier 2012}

\makeatletter
\renewcommand\section{\@startsection{section}{1}{\z@}%
	{2cm \@plus -1ex \@minus -.2ex}%
	{2.3ex \@plus.2ex}%
	{\reset@font\large\bfseries}}
\makeatother

\begin{document}

\maketitle

\begin{center}
\begin{tabular}{lr}
% 
% Présents :
%
  \multicolumn{2}{c}{Présents :} \\ \hline
  \begin{minipage}{0.47\linewidth}
    \vspace{0.1cm}
    \begin{flushleft}
% 	Etudiants :
      Barberot Mathieu \\
      Racenet Joan
    \end{flushleft}
  \end{minipage} &
  \begin{minipage}{0.47\linewidth}
    \vspace{0.1cm}
    \begin{flushright}
% 	Encadrants :
      Bouquet Fabrice \\
      Coqblin Mathias
    \end{flushright}
  \end{minipage} \\
\end{tabular}
\end{center}



\renewcommand{\contentsname}{Ordre du jour :}
\tableofcontents

\section{Découverte du sujet}
Le logiciel BDOvore permet de gérer sa collection de BD grâce à la base de donnée (exhaustive) du site du même nom.\\
L'utilisateur peut ajouter, retirer des BD, récupérer les informations (jaquette, dessinateur, \dots) ou bien faire une liste d'ouvrage à acquérir. Ces manipulations peuvent être faites \textit{en ligne} ou \textit{hors-ligne}. Dans le dernier cas, une base de donnée locale accueille les modifications. Une synchronisation pourra ensuite être effectuée lorsque l'utilisateur sera connecté sur internet.\\
Dans le cadre de ce projet, nous devrons revoir la partie synchronisation de la base de données locale avec la base de données en ligne.\\
Une version du programme nous a été fournie en attendant l'accès au serveur SVN.

\section{Tâches pour la prochaine réunion}
\begin{itemize}
  \item Tester le programme et commencer à regarder le code
  \item Se documenter sur SOAP et sur la synchronisation
  \item Accéder au SVN
\end{itemize}

\section{Prochaine réunion}
La prochaine réunion aura lieu le mercredi 11 janvier 2012 à 16h30 dans le bureau de M. Bouquet.


\end{document}